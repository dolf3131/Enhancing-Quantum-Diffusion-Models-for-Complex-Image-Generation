%  This is free software: you can redistribute it and/or modify
%  it under the terms of the GNU General Public License as published by
%  the Free Software Foundation, either version 3 of the License, or
%  (at your option) any later version.
%
%  This is distributed in the hope that it will be useful,
%  but WITHOUT ANY WARRANTY; without even the implied warranty of
%  MERCHANTABILITY or FITNESS FOR A PARTICULAR PURPOSE.  See the
%  GNU General Public License for more details.
%
%  You can find the GNU General Public License at <http://www.gnu.org/licenses/>.

%  !TEX bib_program = biber
\documentclass[a1paper,portrait]{baposter}
%%%%%%%%%%%%%%%%%%%%%%%%%%%%%%%%%%%%%%%%%%%%%%%%
% Language, Encoding and Fonts
% http://en.wikibooks.org/wiki/LaTeX/Internationalization
%%%%%%%%%%%%%%%%%%%%%%%%%%%%%%%%%%%%%%%%%%%%%%%%
% Select encoding of your inputs. Depends on
% your operating system and its default input
% encoding. Typically, you should use
%   Linux  : utf8 (most modern Linux distributions)
%            latin1 
%   Windows: ansinew
%            latin1 (works in most cases)
%   Mac    : applemac
% Notice that you can manually change the input
% encoding of your files by selecting "save as"
% an select the desired input encoding. 
\usepackage[utf8]{inputenc}
\usepackage[hangul]{kotex}
\usepackage{csquotes}
% Make latex understand and use the typographic
% rules of the language used in the document.
\usepackage[english]{babel}
% Use the vector font Latin Modern which is going
% to be the default font in latex in the future.
\usepackage{helvet}
% Change the default font family from roman to sans serif
\renewcommand{\familydefault}{\sfdefault} % for text
\usepackage[helvet]{sfmath} % for math
% Choose the font encoding
\usepackage[T1]{fontenc}

%%%%%%%%%%%%%%%%%%%%%%%%%%%%%%%%%%%%%%%%%%%%%%%%
% Graphics and Tables
% http://en.wikibooks.org/wiki/LaTeX/Importing_Graphics
% http://en.wikibooks.org/wiki/LaTeX/Tables
% http://pgfplots.sourceforge.net/
%%%%%%%%%%%%%%%%%%%%%%%%%%%%%%%%%%%%%%%%%%%%%%%%
% You cannot use floats in the baposter theme.
% We therefore load the caption package which provides
% the command \captionof
% Set up how figure and table captions are displayed
\usepackage{caption}
\captionsetup{
  font=small,% set font size to footnotesize
  labelfont=bf % bold label (e.g., Figure 3.2) font
}
% Make the standard latex tables look so much better
\usepackage{array,booktabs}
% For creating beautiful plots
\usepackage{pgfplots}

%%%%%%%%%%%%%%%%%%%%%%%%%%%%%%%%%%%%%%%%%%%%%%%%
% Mathematics
% http://en.wikibooks.org/wiki/LaTeX/Mathematics
%%%%%%%%%%%%%%%%%%%%%%%%%%%%%%%%%%%%%%%%%%%%%%%%
% Defines new environments such as equation,
% align and split 
\usepackage{amsmath}
% Adds new math symbols
\usepackage{amssymb}

%%%%%%%%%%%%%%%%%%%%%%%%%%%%%%%%%%%%%%%%%%%%%%%%
% Colours
% http://en.wikibooks.org/wiki/LaTeX/Colors
%%%%%%%%%%%%%%%%%%%%%%%%%%%%%%%%%%%%%%%%%%%%%%%%
\selectcolormodel{RGB}
% define the three blue colors
\definecolor{nustblue}{RGB}{0,70,120}% dark blue
\definecolor{nustblue1}{RGB}{112,110,140} % light blue
\definecolor{nustblue2}{RGB}{190,190,200} % lighter blue
\definecolor{nustblue3}{RGB}{100,100,255}
%%%%%%%%%%%%%%%%%%%%%%%%%%%%%%%%%%%%%%%%%%%%%%%%
% Lists
% http://en.wikibooks.org/wiki/LaTeX/List_Structures
%%%%%%%%%%%%%%%%%%%%%%%%%%%%%%%%%%%%%%%%%%%%%%%%
% Easier configuration of lists
\usepackage{enumitem}
%configure itemize
\setlist{%
  topsep=0pt,% set space before and after list
  noitemsep,% remove space between items
  labelindent=\parindent,% set the label indentation to the paragraph indentation
  leftmargin=*,% remove the left margin
  font=\color{nustblue}\normalfont, %set the colour of all bullets, numbers and descriptions to nustblue
}
% use set<itemize,enumerate,description> if you have an older latex distribution
\setitemize[1]{label={\raise1.25pt\hbox{$\blacktriangleright$}}}
\setitemize[2]{label={\scriptsize\raise1.25pt\hbox{$\blacktriangleright$}}}
\setitemize[3]{label={\raise1.25pt\hbox{$\star$}}}
\setitemize[4]{label={-}}
%\setenumerate[1]{label={\theenumi.}}
%\setenumerate[2]{label={(\theenumii)}}
%\setenumerate[3]{label={\theenumiii.}}
%\setenumerate[4]{label={\theenumiv.}}
%\setdescription{font=\color{nustblue}\normalfont\bfseries}

% use setlist[<itemize,enumerate,description>,<level>] if you have a newer latex distribution
%\setlist[itemize,1]{label={\raise1.25pt\hbox{$\blacktriangleright$}}}
%\setlist[itemize,2]{label={\scriptsize\raise1.25pt\hbox{$\blacktriangleright$}}}
%\setlist[itemize,3]{label={\raise1.25pt\hbox{$\star$}}}
%\setlist[itemize,4]{label={-}}
%\setlist[enumerate,1]{label={\theenumi.}}
%\setlist[enumerate,2]{label={(\theenumii)}}
%\setlist[enumerate,3]{label={\theenumiii.}}
%\setlist[enumerate,4]{label={\theenumiv.}}
%\setlist[description]{font=\color{nustblue}\normalfont\bfseries}

%%%%%%%%%%%%%%%%%%%%%%%%%%%%%%%%%%%%%%%%%%%%%%%%
% Misc
%%%%%%%%%%%%%%%%%%%%%%%%%%%%%%%%%%%%%%%%%%%%%%%%
% change/remove some names
\addto{\captionsenglish}{
  %remove the title of the bibliograhpy
  \renewcommand{\refname}{\vspace{-0.7em}}
  %change Figure to Fig. in figure captions
  \renewcommand{\figurename}{Fig.}
}
% create links
\usepackage{url}
%note that the hyperref package is currently incompatible with the baposter class



%%%%%%%%%%%%%%%%%%%%%%%%%%%%%%%%%%%%%%%%%%%%%%%%
% Macros
%%%%%%%%%%%%%%%%%%%%%%%%%%%%%%%%%%%%%%%%%%%%%%%%
\newcommand{\alert}[1]{{\color{nustblue}#1}}




\usepackage{graphicx}
\usepackage{tikz}
\usepackage{quantikz}
\usepackage{pgfplots}
\usepackage{amsmath}
\usepackage{amssymb}
\usepackage{braket}
\usepackage{multicol}
\usepackage{float}

\usepackage[backend=biber,style=numeric,url=true]{biblatex}
\addbibresource{bibliography.bib}

\usepackage{hyperref}
\usepackage{xurl} 
\hypersetup{
    colorlinks=true,
    linkcolor=nustblue, 
    urlcolor=nustblue,  
    citecolor=nustblue,
    breaklinks=true,   
    pdfpagemode=UseNone
}



\usepackage{setspace}
\setstretch{1.12}
\setlength{\parskip}{0pt}


%%%%%%%%%%%%%%%%%%%%%%%%%%%%%%%%%%%%%%%%%%%%%%%%
% Document Start 
%%%%%%%%%%%%%%%%%%%%%%%%%%%%%%%%%%%%%%%%%%%%%%%%
\pgfplotsset{compat=1.18}
\begin{document}
%%%%%%%%%%%%%%%%%%%%%%%%%%%%%%%%%%%%%%%%%%%%%%%%
% Some changes that cannot be made in the preamble
%%%%%%%%%%%%%%%%%%%%%%%%%%%%%%%%%%%%%%%%%%%%%%%%
% set the background of the poster
\background{
  \begin{tikzpicture}[remember picture,overlay]%
    %the poster background color
    \fill[fill=nustblue2] (current page.north west) rectangle (current page.south east);
    %the header
    \fill [fill=nustblue] (current page.north west) rectangle ([yshift=-\headerheight] current page.north east);
  \end{tikzpicture}
}
% if you want to reduce the space before and after equations, use and adjust
% the following lines
%\addtolength{\abovedisplayskip}{-2mm}
%\addtolength{\belowdisplayskip}{-2mm}

%%%%%%%%%%%%%%%%%%%%%%%%%%%%%%%%%%%%%%%%%%%%%%%%
% General poster setup
%%%%%%%%%%%%%%%%%%%%%%%%%%%%%%%%%%%%%%%%%%%%%%%%
\begin{poster}{
  %general options for the poster
  grid=false,
  columns=3,
%  colspacing=4.2mm,
  headerheight=0.1\textheight,
  background=none,
%  bgColorOne=red!42, %is used when background != user and none
%  bgColortwo=green!42, %is used when background is shaded
  eyecatcher=true,
  %posterbox options
  headerborder=closed,
  borderColor=nustblue,
  headershape=rectangle,
  headershade=plain,
  headerColorOne=nustblue,
%  headerColortwo=yellow!42, %is used when the header background is shaded
  textborder=rectangle,
  boxshade=plain,
  boxColorOne=white,
%  boxColorTwo=cyan!42,%is used when the text background is shaded
  headerFontColor=white,
  headerfont=\Large\sf\bf,
  linewidth=1pt
}
%the Eye Catcher (the logo on the left)
{
  %this can be commented out or replaced by a company/department logo
  \begin{tabular}{c}
  \includegraphics[height=0.80\headerheight]{logos/qiskit.png} \\
  \includegraphics[height=0.20\headerheight]{logos/IBM-Quantum.png}
  \end{tabular}
}
%the poster title
{
  {\color{nustblue} \bf \huge Enhancing Quantum Diffusion Model for Complex Image Generation} \\
  \vspace{0.3em}
  {\color{nustblue} \Large \it Qiskit Advocate Mentorship Program: 33}
}
%the author(s)
{\small
  \vspace{1em}
  Jeongbin Jo$^1$, Santanam Wishal, Shah Md Khalil Ullah, Shan Kowalski \\
  \vspace{0.3em}
  $^1$\textit{Dept.\ of Physics, Yonsei University},
  \texttt{jeongbin033@yonsei.ac.kr}
}
%the logo (the logo on the right)
{
  %this can be commented out or replaced by a conference logo
  \includegraphics[height=0.75\headerheight]{logos/qiskit.png}
}

%%%%%%%%%%%%%%%%%%%%%%%%%%%%%%%%%%%%%%%%%%%%%%%%
% the actual content of the poster begins here
%%%%%%%%%%%%%%%%%%%%%%%%%%%%%%%%%%%%%%%%%%%%%%%%

\begin{posterbox}[name=intro,column=0,row=0]{Introduction}
\textbf{Qiskit Advocate Mentorship Program (QAMP)} is a program focused on bring new contributors into Qiskit open source software development where Qiskit advocates work on a 3-month projects under the guidance of mentors. It is an initiative within the \href{https://www.ibm.com/quantum/qiskit#advocates}{Qiskit advocate program} designed to support growth and collaboration within our vibrant community.

Access the QAMP information deck \href{https://github.com/qiskit-advocate/qamp-2025/blob/main/QAMP_call.pdf}{here}.
\end{posterbox}

\begin{posterbox}[name=encoding,column=0,below=intro]{Quantum Data Encoding}
\begin{description}
  \item[Basis Encoding]:
    \begin{equation*}
      \ket{x} = \ket{b_1, b_2, \dots, b_P}, \quad b_i \in \{0, 1\}
    \end{equation*}

  \item[Amplitude Encoding]:
    $n = \lceil \log_2 N \rceil$.

    \begin{equation*}
      \ket{\psi_x} = \frac{1}{\alpha}\sum_{i=1}^N x_i \ket{i}
    \end{equation*}
    
    \begin{equation*}
      \alpha = \sqrt{\sum_{i=1}^{N}|x_i|^2}
    \end{equation*}
    
  \item[Angle Encoding]:
    \begin{equation*}
      \ket{\vec{x}} = \bigotimes_{k=1}^N R_Y(x_k)\ket{0} = 
      \bigotimes_{k=1}^N \left( \cos\left(\frac{x_k}{2}\right)\ket{0} + \sin\left(\frac{x_k}{2}\right)\ket{1} \right)
    \end{equation*} 
    
  \item[Phase Encoding]:
    \begin{equation*}
      \ket{\vec{x}} = \bigotimes_{k=1}^N P(x_k)\ket{+} = 
      \frac{1}{\sqrt{2^N}} \bigotimes_{k=1}^N \left( \ket{0} + e^{i x_k}\ket{1} \right)  
    \end{equation*}
    
  \item[Dense Angle Encoding]:
    \begin{equation*}
      \ket{x_k, x_l} = R_Z(x_l) R_Y(x_k)\ket{0} = \cos\left(\frac{x_k}{2}\right)\ket{0} + e^{i x_l} \sin\left(\frac{x_k}{2}\right)\ket{1} 
    \end{equation*}

    \begin{equation*}
      \ket{\vec{x}} = \bigotimes_{k=1}^{N/2}\left( \cos{x_{2k-1}}\ket{0} + e^{ix_{2k}}\sin{x_{2k-1}\ket{1}} \right)
    \end{equation*}

\end{description}

\end{posterbox}

\begin{posterbox}[name=ansatz,column=0,below=encoding,above=bottom]{Parameterized Quantum Circuit}

\begin{description}
  \item[Optimal two-qubit decomposition]: 
  
  \textbf{Vatan-Williams} decomposition \cite{Vatan_2004}.
  The circuit structure, depicted in Fig.~\ref{fig:interaction_block}, consists of three CNOT gates interleaved with parameterized single-qubit rotations. 
  This configuration is capable of simulating the unitary operator $U = \exp(-i(\alpha \sigma_x \otimes \sigma_x + \beta \sigma_y \otimes \sigma_y + \gamma \sigma_z \otimes \sigma_z))$ up to local basis transformations. \\
\end{description}

\begin{minipage}{\linewidth}
  \centering
  \begin{quantikz}
    \lstick{$q_0$} & \qw & \targ{} & \gate{R_z(\frac{\pi}{2}-2\gamma)} & \ctrl{1} & \qw & \targ{} & \gate{R_z(\frac{\pi}{2})} & \qw \\
    \lstick{$q_1$} & \gate{R_z(-\frac{\pi}{2})} & \ctrl{-1} & \gate{R_y(2\alpha-\frac{\pi}{2})} & \targ{} & \gate{R_y(\frac{\pi}{2}-2\beta)} & \ctrl{-1} & \qw & \qw
  \end{quantikz}
  \captionof{figure}{This structure implements the optimal decomposition for arbitrary two-qubit entangling gates, characterized by the parameters $\alpha, \beta, \gamma$. Note the alternating CNOT direction and the specific rotation angles derived from the canonical decomposition.}
  \label{fig:interaction_block}
\end{minipage}



\end{posterbox}


\begin{posterbox}[name=diffusion,span=2,column=1,row=0]{Diffusion Process}
  While classical diffusion models operate on probability distributions over classical data, 
  Quantum Diffusion Models (QDMs) extend this concept to the Hilbert space of quantum states. 
  The goal is to generate quantum states (density matrices) from a maximally mixed state.

  
  \textbf{Forward Process: Depolarizing Channel}
  Instead of adding Gaussian noise, the forward process in QDM is typically modeled as a standard depolarizing channel acting on a density matrix $\rho$. 
  For a system of $n$ qubits with dimension $d=2^n$, the state at step $t$ is given by:

  \begin{equation}
    \rho_t = \mathcal{E}_t(\rho_{t-1}) = (1 - p_t)\rho_{t-1} + p_t \frac{I}{d}
  \end{equation}

  where $p_t \in [0, 1]$ is the depolarization probability (noise schedule). 
  Similar to the classical case, we can express $\rho_t$ directly from the initial state $\rho_0$. 
  Let $\alpha_t = \prod_{s=1}^t (1 - p_s)$, then:

  \begin{equation}
    \rho_t = \alpha_t \rho_0 + (1 - \alpha_t) \frac{I}{d}
  \end{equation}

  As $t \to T$, $\alpha_T \to 0$, and the state converges to the maximally mixed state $\rho_T \approx \frac{I}{d}$, 
  which contains no information about $\rho_0$.

  \begin{minipage}{\linewidth}
    \centering
    \includegraphics[width=\linewidth]{images/forward.png}   
    \captionof{figure}{Forward process}
    \label{fig:forward}
  \end{minipage}
\end{posterbox}


\begin{posterbox}[name=denoising,column=1,below=diffusion]{Denoising Process}
  \textbf{Reverse Process: Quantum Denoising}
  The reverse process aims to restore the quantum state from the noise. 
  This is modeled by a parameterized quantum circuit (PQC), denoted as a unitary operator $U(\theta)$. 
  The discrete reverse step can be approximated as:

  \begin{equation}
    \rho_{t-1} \approx \mathcal{D}_\theta(\rho_t, t) = U(\theta_t) \rho_t U^\dagger(\theta_t)
  \end{equation}

  For more complex generative tasks, the reverse process may involve ancillary qubits and measurements to simulate non-unitary maps.

  \textbf{Loss Function: Fidelity or Trace Distance}
  Quantum models often use Fidelity or Hilbert-Schmidt distance to measure the closeness between the generated state and the target state. 
  A common objective is to maximize the overlap with the target pure state $\ket{\psi}$:

  \begin{equation}
    \mathcal{L}(\theta) = 1 - \mathbb{E} \left[ \bra{\psi} \rho_{\text{gen}}(\theta) \ket{\psi} \right]
  \end{equation}
  Alternatively, for density matrices, we minimize the trace distance or maximize the quantum fidelity $F(\rho, \sigma) = (\text{tr}\sqrt{\sqrt{\rho}\sigma\sqrt{\rho}})^2$.

  \begin{minipage}{\linewidth}
    \centering
    \includegraphics[width=0.55\linewidth]{images/sample.png}
    \captionof{figure}{Sample of generation process in random time t.}
    \label{fig:sample}
  \end{minipage}
\end{posterbox}



\begin{posterbox}[name=architecture,column=1,below=denoising,above=bottom]{Architecture}
We adopted bottleneck architecture Fig.~\ref{fig:quantumunet}, that commonly used for convolutional neural network(CNN) in machine learning.

\begin{minipage}{\linewidth}
  \centering
  \includegraphics[width=\linewidth]{images/QuantumUNet_Final_LR.png}
  \captionof{figure}{Quantum Unet architecture}
  \label{fig:quantumunet}
\end{minipage}
\end{posterbox}






\begin{posterbox}[name=measurement,column=2,below=diffusion]{Measurement}
\begin{itemize}
  \item \textbf{ANO Measurement}
  
  \item \textbf{Expressibility based on Meyer-Wallach values} \cite{Meyer_2002}
  For a given state $|\psi\rangle$, the MW measure is defined as the average purity of the single-qubit reduced density matrices:
  \begin{equation}
      Q(|\psi\rangle) = \frac{4}{n} \sum_{k=1}^{n} D(\iota_k(|\psi\rangle)) = \frac{4}{n} \sum_{k=1}^{n} \frac{1}{2} \left( 1 - \text{Tr}(\rho_k^2) \right)
  \end{equation}
  where $\rho_k = \text{Tr}_{\neg k}(|\psi\rangle\langle\psi|)$ is the reduced density matrix of the $k$-th qubit. The value $Q$ ranges from 0 (product states, unentangled) to 1 (maximally entangled, e.g., GHZ states).

  We estimate the \textit{Entangling Capability} \cite{Sim_2019} of our ansatz by averaging $Q$ over an ensemble of states sampled with uniformly random parameters:
  \begin{equation}
      \bar{Q} = \frac{1}{S} \sum_{i=1}^{S} Q(|\psi(\theta_i)\rangle)
  \end{equation}
\end{itemize}
\begin{minipage}{\linewidth}
  \centering
  \includegraphics[width=\linewidth]{images/expressibility.png}
  \captionof{figure}{Visualization of the single-qubit reduced state distribution on the Bloch sphere generated by our ansatz with ANO.}
  \label{fig:expressibility}
\end{minipage}

\end{posterbox}




\begin{posterbox}[name=post,column=2,below=measurement]{Post Project}


\end{posterbox}


\begin{posterbox}[name=refs,column=2,below=post,above=bottom]{References}

\begingroup
%\raggedright
\setlength\bibitemsep{0.2em}                 % 항목 간 간격 줄이기
\setlength\baselineskip{0.4\baselineskip}   % 전체 줄 간격 줄이기
\renewcommand*{\bibfont}{\tiny}             % 글자 크기도 소형화
\printbibliography[heading=none]
\endgroup

\end{posterbox}
\end{poster}
\end{document}
